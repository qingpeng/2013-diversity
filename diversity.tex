% Template for PLoS
% Version 1.0 January 2009
%
% To compile to pdf, run:
% latex plos.template
% bibtex plos.template
% latex plos.template
% latex plos.template
% dvipdf plos.template

\documentclass[10pt]{article}

% amsmath package, useful for mathematical formulas
\usepackage{amsmath}
% amssymb package, useful for mathematical symbols
\usepackage{amssymb}

% graphicx package, useful for including eps and pdf graphics
% include graphics with the command \includegraphics
\usepackage{graphicx}

% cite package, to clean up citations in the main text. Do not remove.
\usepackage{cite}

\usepackage{color} 

% Use doublespacing - comment out for single spacing
%\usepackage{setspace} 
%\doublespacing


% Text layout
\topmargin 0.0cm
\oddsidemargin 0.5cm
\evensidemargin 0.5cm
\textwidth 16cm 
\textheight 21cm

% Bold the 'Figure #' in the caption and separate it with a period
% Captions will be left justified
\usepackage[labelfont=bf,labelsep=period,justification=raggedright]{caption}

% Use the PLoS provided bibtex style
\bibliographystyle{plos2009}

% Remove brackets from numbering in List of References
\makeatletter
\renewcommand{\@biblabel}[1]{\quad#1.}
\makeatother


% Leave date blank
\date{}

\pagestyle{myheadings}
%% ** EDIT HERE **


%% ** EDIT HERE **
%% PLEASE INCLUDE ALL MACROS BELOW

%% END MACROS SECTION

\begin{document}

% Title must be 150 characters or less
\begin{flushleft}
{\Large
\textbf{Diversity Estimation of Metagenomics Samples}
}
% Insert Author names, affiliations and corresponding author email.
%\\
%Qingpeng Zhang$^{1}$, 
%Jason Pell$^{1}$,
%Rosangela Canino-Koning$^{1}$,
%Adina Chuang Howe$^{2,3}$,
%C. Titus Brown$^{1,2\ast}$
%\\
%\bf{1} Computer Science and Engineering, Michigan State University,
%East Lansing, MI, USA
%\\
%\bf{2} Microbiology and Molecular Genetics, Michigan State University,
%East Lansing, MI, USA
%\\
%\bf{3} Plant, Soil, and Microbial Sciences, Michigan State University, 
%East Lansing, MI, USA
%\\
%$\ast$ E-mail: ctb@msu.edu
\end{flushleft}


\section*{Abstract}



% Please keep the Author Summary between 150 and 200 words
% Use first person. PLoS ONE authors please skip this step. 
% Author Summary not valid for PLoS ONE submissions.   
\section*{Author Summary}

\section*{Introduction}
Comparison of metagenomics samples
\\
Coverage estimation of metagenomics reads
\\
Diversity evaluation of metagenomics samples

reference-free, assembly-free annotation-free, binning-free

% Results and Discussion can be combined.
\section*{Results}

\subsection*{Comparison of metagenomics samples}

\subsubsection*{Theoretical analysis}
\subsubsection*{Synthetic data}
same number of species (100), different composition
\\
same coverage( 20X, 1X)
\\
same error rate ( no error, illumina error profile)
\\
Coverage matters, as expected
\\
after saturation, it can give correct number. if too low coverage, it is not accurate.
But there should be a way to figure out the relationship.
\\
with 1X coverage, 50\% of real coverage.

\\next to do:
\\
1. figure out the relationship between coverage and overlap accuracy
\\
2. synthetic data with real bacterial genomes.
\\
3. 

\section*{Discussion}


%\subsection*{TBD}


\section*{Methods}

\subsection*{Code and data set availability}

\subsection*{synthetic data}
We built 4 series of synthetic data sets:\\
Each series include four sampels with specific composition:\\
SampleA: 100 species with 80 common to B\\
SampleB: 100 species with 80 common to A\\
SampleC: 100 species with 20 common to A/B, and 60 common to D\\
SampleD: 100 species with 20 common to A/B, and 60 common to D\\

4 Series with different coverage and different error rate:\\
1. high coverage(20X) without error\\
2. low coverage(1X) without error\\
3. high coverage(20X) with error, illumina error profile\\
3. low coverage(1X) without error, illumina error profile\\
% @CTB update
%
%The version of khmer used to generate the results below is available
%at http://github.com/ged-lab/khmer.git, tag '2013-khmer-counting'.
%Scripts specific to this paper are available in the paper repository
%at https://github.com/ged-lab/2013-khmer-counting.
%The iPython\cite{4160251} notebook file and data analysis to generate the figures are also
%available in that github repository. 


%\section*{References}
% The bibtex filename
%\bibliography{khmer-counting}

\section*{Figure Legends}

%\begin{figure}[!ht]
%\begin{center}
%%\includegraphics[width=4in]{figure_name.2.eps}
%\end{center}
%\caption{
%{\bf Bold the first sentence.}  Rest of figure 2  caption.  Caption 
%should be left justified, as specified by the options to the caption 
%package.
%}
%\label{Figure_label}
%\end{figure}


\graphicspath{./figure/}

% @CTB explain why different runs
\begin{figure}[!ht]
\centerline{\includegraphics[width=8in]{./figure/figure1}}
\caption{\bf \% of reads in sampleX are covered in sampleY }
\label{fig:figure1}
\end{figure}


\section*{Tables}
%\begin{table}[!ht]
%\caption{
%\bf{Table title}}
%\begin{tabular}{|c|c|c|}
%table information
%\end{tabular}
%\begin{flushleft}Table caption
%\end{flushleft}
%\label{tab:label}
% \end{table}

% @CTB fix data set names/descr; caption; reference.
%\begin{table}[!ht]
%\caption{
%\bf{Benchmark soil metagenome data sets for k-mer counting performance, taken from
%\cite{Howe2012}.}}
%\begin{tabular}{ |c | c |c| c|c| }
%Data set & size of file (GB) & number of reads & number of distinct
%k-mers & total number of k-mers \\
%\hline \\
%subset 1        & 1.90 &  9,744,399 &   561,178,082 &   630,207,985 \\
%subset 2        & 2.17 & 19,488,798 & 1,060,354,144 & 1,259,079,821 \\
%subset 3        & 3.14 & 29,233,197 & 1,445,923,389 & 1,771,614,378 \\
%subset 4        & 4.05 & 38,977,596 & 1,770,589,216 & 2,227,756,662 \\
%entire data set & 5.00 & 48,721,995 & 2,121,474,237 & 2,743,130,683 \\
%\end{tabular}
%\begin{flushleft}
%\end{flushleft}
%\label{table:datasets}
%\end{table}
%


\end{document}

